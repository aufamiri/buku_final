% Mengubah keterangan `Abstract` ke bahasa indonesia.
% Hapus bagian ini untuk mengembalikan ke format awal.
\renewcommand\abstractname{Abstrak}

\begin{abstract}

  % Ubah paragraf berikut sesuai dengan abstrak dari penelitian.
  Berita palsu atau yang biasa disebut hoaks adalah suatu yang hal yang sering melanda Indonesia. Dengan adanya sosial media, suatu berita palsu dapat memiliki tingkat penyebaran yang sangat luas. Selain itu, masyarakat Indonesia memiliki tingkat kecenderungan untuk menyebarkan berita palsu yang cukup tinggi. Sehingga, suatu metode pendeteksi berita palsu harus ada. Penelitian ini memanfaatkan algoritma BERT yang digunakan untuk mendeteksi apakah suatu berita adalah berita hoaks atau tidak secara otomatis. Dari suatu teks yang mentah, akan dilakukan tokenisasi sebelum akhirnya dimasukkan ke dalam algoritma BERT. Selanjutnya, keluaran dari BERT akan dijadikan sebagai inputan dari algoritma klasifikasi Linear Regression.  Barulah pada saat ini, kita bisa mendapatkan klasifikasi apakah suatu teks itu berupa berita hoaks atau tidak. Tujuan dari penelitian ini adalah untuk membuat sebuah model yang dapat digunakan untuk melakukan klasifikasi suatu teks apakah termasuk ke dalam berita hoaks atau tidak.

\end{abstract}

% Mengubah keterangan `Index terms` ke bahasa indonesia.
% Hapus bagian ini untuk mengembalikan ke format awal.
\renewcommand\IEEEkeywordsname{Kata kunci}

\begin{IEEEkeywords}

  % Ubah kata-kata berikut sesuai dengan kata kunci dari penelitian.
  BERT, Hoaks, Klasifikasi, Linear Regression

\end{IEEEkeywords}
