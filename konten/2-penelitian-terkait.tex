% Ubah judul dan label berikut sesuai dengan yang diinginkan.
\section{Penelitian Terkait}
\label{sec:penelitianterkait}

Sudah terdapat beberapa penelitian yang pernah dilakukan oleh orang lain mengenai pendeteksi berita hoaks ini. Aggarway et al. pernah melakukan penelitian untuk membandingkan antara BERT, XGBoost dan LSTM untuk melakukan klasifikasi berita palsu berbahasa inggris. Dari penelitian tersebut didapatkan bahwa BERT memiliki tingkat akurasi yang lebih tinggi apabila dibandingkan dengan XGBoost dan LSTM \cite{bert_news_classi}. Bahad et al. melakukan penelitian yang membandingkan antara CNN, RNN, \textit{uni-directional} LSTM RNN dan \textit{bi-directional} LSTM RNN. Hasil dari penelitian tersebut menunjukkan bahwa penggunaan LSTM ditambah dengan \textit{attention} baik itu \textit{uni-directional} maupun \textit{bi-directional} memiliki tingkat akurasi yang lebih tinggi apabila dibandingkan dengan CNN atau RNN \cite{bahad_lstm}. Dari kedua penelitian tersebut, dapat diambil kesimpulan bahwa algoritma yang 'mengingat' atau mengetahui suatu konteks dalam teks akan memiliki tingkat akurasi yang lebih tinggi dibanding algoritma dengan pendekatan yang lain.

Untuk penelitian pendeteksi berita hoaks dengan berbahasa indonesia, terdapat beberapa penelitian yang pernah dilakukan seperti oleh Prasetijo et al. yang meneliti penggunaan SVM dan SGD untuk mendeteksi berita hoaks berbahasa indonesia. Penelitian tersebut berhasil membuat suatu model dengan tingkat akurasi sebesar 85\% \cite{prasetijo}. Penelitian lain yang dilakukan oleh Rahutomo et al. dengan menggunakan algoritma \textit{naive bayes} berhasil menghasilkan akurasi sebesar 80\% \cite{rahutomo}.
