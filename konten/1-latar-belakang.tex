% Ubah judul dan label berikut sesuai dengan yang diinginkan.
\section{Latar Belakang}
\label{sec:latarbelakang}

% Ubah paragraf-paragraf pada bagian ini sesuai dengan yang diinginkan.
Berita adalah laporan atau cerita faktual yang disajikan paling cepat, memiliki pemaparan masalah yang baik, serta berlaku adil kepada seluruh masalah yang disajikan \cite{rani2013persepsi}. Berita memiliki peran yang sangat penting dalam masyarakat karena sebagai media yang dapat digunakan untuk mengetahui peristiwa paling baru, juga dapat digunakan sebagai media untuk menambah wawasan.

Hoaks atau berita palsu adalah sebuah cara atau usaha yang berusaha untuk menipu orang sehingga mempercayai sesuai yang salah sebagai hal benar dan seringnya hal yang salah tersebut sama sekali tidak masuk akal \cite{berita_bohong}. Selain kerugian dalam hal pengetahuan, berita palsu memiliki efek yang beragam, seperti kerugian dalam bentuk reputasi, harta benda, sampai ancaman pembunuhan.

Berdasarkan data yang diperoleh dari Kementrian Komunikasi dan Informatika total jumlah berita palsu yang ditemukan pada tahun Agustus 2018 sampai dengan Maret 2020 berjumlah 5156. pada bulan Januari 2020 sampai Maret 2020, sudah terdapat 959 berita palsu yang ditemukan \cite{kominfoStatHoax}. Masih dari sumber yang sama, pada bulan Juni 2020, hampir setiap harinya ditemukan puluhan berita palsu baru \cite{kominfoJuni2020}.

Berita hoaks juga memiliki tingkat penyebaran yang cepat seiring dengan semakin tingginya penggunaan media sosial oleh masyarakat. Berdasarkan survey yang dilakukan oleh Khan dan Idris, lebih dari 50\% masyarakat Indonesia memiliki tingkat kecenderungan untuk melakukan share suatu tautan berita tanpa melakukan validasi terlebih dahulu \cite{khan}. Survey lain yang dilakukan oleh Kunto yang melibatkan 480 responden di Kota Jawa Barat menunjukkan bahwa sekitar 30\% masyarakat Jawa Barat memiliki kecenderungan menengah sampai tinggi untuk menyebarkan berita palsu \cite{kuntoUmur}. Dari sampel tersebut, dapat disimpulkan bahwa Indonesia memiliki kecenderungan tinggi untuk menyebarkan berita palsu.

\textit{Neural Networks} adalah salah satu cabang dalam pembelajaran mesin yang menerapkan \textit{neurons} layaknya struktur otak manusia untuk memproses data dan menghasilkan keluaran. Salah satu metode \textit{neural network} yang cukup baru adalah \textit{Bi-directional Encoder Representations from Transformers} atau disingkat sebagai BERT. BERT adalah metode yang digunakan untuk mendapatkan suatu konteks dalam suatu teks yang dimasukkan.

Pembahasan pada paper ini dimulai dengan presentasi mengenai penelitian lain (Bagian \ref{sec:penelitianterkait}).
Kemudian dilanjutkan dengan penjelasan mengenai desain dan implementasi dari sistem yang dibuat (Bagian \ref{sec:desainimplementasi}).
Berdasarkan hal tersebut, kami menunjukkan lorem ipsum (Bagian \ref{sec:loremipsum}).
Terakhir, didapatkan kesimpulan dari penelitian yang telah dilakukan (Bagian \ref{sec:kesimpulan}).
