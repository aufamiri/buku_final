% Ubah judul dan label berikut sesuai dengan yang diinginkan.
\section{Latar Belakang}
\label{sec:latarbelakang}

% Ubah paragraf-paragraf pada bagian ini sesuai dengan yang diinginkan.
Berita adalah laporan atau cerita faktual yang disajikan paling cepat, memiliki pemaparan masalah yang baik, serta berlaku adil kepada seluruh masalah yang disajikan \cite{rani2013persepsi}. Berita memiliki peran yang sangat penting dalam masyarakat karena sebagai media yang dapat digunakan untuk mengetahui peristiwa paling baru, juga dapat digunakan sebagai media untuk menambah wawasan.

Hoaks atau berita palsu adalah sebuah cara atau usaha yang berusaha untuk menipu orang sehingga mempercayai sesuai yang salah sebagai hal benar dan seringnya hal yang salah tersebut sama sekali tidak masuk akal \cite{berita_bohong}. Selain kerugian dalam hal pengetahuan, berita palsu memiliki efek yang beragam, seperti kerugian dalam bentuk reputasi, harta benda, sampai ancaman pembunuhan.

Berdasarkan data yang diperoleh dari Kementrian Komunikasi dan Informatika,  total jumlah berita palsu yang ditemukan pada tahun Agustus 2018 sampai dengan Maret 2020 berjumlah 5156. pada bulan Januari 2020 sampai Maret 2020, sudah terdapat 959 berita palsu yang ditemukan \cite{kominfoStatHoax}. Masih dari sumber yang sama, pada bulan Juni 2020, hampir setiap harinya ditemukan puluhan berita palsu baru \cite{kominfoJuni2020}.

Berita hoaks juga memiliki tingkat penyebaran yang cepat seiring dengan semakin tingginya penggunaan media sosial oleh masyarakat. Berdasarkan survey yang dilakukan oleh Khan dan Idris, lebih dari 50\% masyarakat Indonesia memiliki tingkat kecenderungan untuk melakukan share suatu tautan berita tanpa melakukan validasi terlebih dahulu \cite{khan}. Survey lain yang dilakukan oleh Kunto dengan melibatkan 480 responden di Kota Jawa Barat menunjukkan bahwa sekitar 30\% masyarakat Jawa Barat memiliki kecenderungan menengah sampai tinggi untuk menyebarkan berita palsu \cite{kuntoUmur}. Dari sampel tersebut, dapat disimpulkan bahwa Indonesia memiliki kecenderungan tinggi untuk menyebarkan berita palsu.

\textit{Neural Networks} adalah salah satu cabang dalam pembelajaran mesin yang menerapkan \textit{neurons} layaknya struktur otak manusia untuk memproses data dan menghasilkan keluaran. Salah satu metode \textit{neural network} yang cukup baru adalah \textit{Bi-directional Encoder Representations from Transformers} atau disingkat sebagai BERT. BERT adalah metode yang digunakan untuk mendapatkan suatu konteks dalam suatu teks yang dimasukkan.

Sudah terdapat beberapa penelitian yang pernah dilakukan oleh orang lain mengenai pendeteksi berita hoaks ini. Aggarway et al. pernah melakukan penelitian untuk membandingkan antara BERT, XGBoost dan LSTM untuk melakukan klasifikasi berita palsu berbahasa inggris. Dari penelitian tersebut didapatkan bahwa BERT memiliki tingkat akurasi yang lebih tinggi apabila dibandingkan dengan XGBoost dan LSTM \cite{bert_news_classi}. Bahad et al. melakukan penelitian yang membandingkan antara CNN, RNN, \textit{uni-directional} LSTM RNN dan \textit{bi-directional} LSTM RNN. Hasil dari penelitian tersebut menunjukkan bahwa penggunaan LSTM ditambah dengan \textit{attention} baik itu \textit{uni-directional} maupun \textit{bi-directional} memiliki tingkat akurasi yang lebih tinggi apabila dibandingkan dengan CNN atau RNN \cite{bahad_lstm}. Dari kedua penelitian tersebut, dapat diambil kesimpulan bahwa algoritma yang 'mengingat' atau mengetahui suatu konteks dalam teks akan memiliki tingkat akurasi yang lebih tinggi dibanding algoritma dengan pendekatan yang lain.

Untuk penelitian pendeteksi berita hoaks dengan berbahasa indonesia, terdapat beberapa penelitian yang pernah dilakukan seperti oleh Prasetijo et al. yang meneliti penggunaan SVM dan SGD untuk mendeteksi berita hoaks berbahasa indonesia. Penelitian tersebut berhasil membuat suatu model dengan tingkat akurasi sebesar 85\% \cite{prasetijo}. Penelitian lain yang dilakukan oleh Rahutomo et al. dengan menggunakan algoritma \textit{naive bayes} berhasil menghasilkan akurasi sebesar 80\% \cite{rahutomo}.

Tujuan dari penelitian ini adalah pengembangan pendeteksi berita palsu berbahasa indonesia dengan menggunakan BERT yang diharapakan dapat membantu meningkatkan tingkat efisiensi dan akurasi pendeteksi berita palsu berbahasa indonesia.
